\documentclass[11pt]{article}

\usepackage{bera}
\usepackage{url}
\setlength{\parindent}{0cm}

\begin{document}
\title{LWC 2013 - MPS}
\author{Remi Bosman, Klemens Schindler, Albert Gerritsen\\
Eugen Schindler, Martin Palatnik}

\maketitle

\section{Case introduction}
In this tutorial we describe how to build a language to design questionnaires. The name of this language is Questionnaire Language, or QL.
More details on the requirements of the language can be found here:

\url{http://www.languageworkbenches.net/images/5/53/Ql.pdf}

\section{Getting started}
Download JetBrains MPS for your platform here:
\url{http://www.jetbrains.com/mps/download/index.html}
Note: This tutorial was made using MPS 2.5.

For the advanced editor features we will be adding later during this tutorial, the following plugins are required:
\begin{description}
\item[MPS Editor Override] \url{http://github.com/slisson/mps-editor-override}
\item[MPS Multiline Property Editor Component] \url{http://github.com/slisson/mps-multiline}
\item[MPS Richtext Editor Component] \url{https://github.com/slisson/mps-richtext}
\end{description}

Plugins can be installed as follows:
\begin{enumerate}
\item Start MPS
\item Click \textit{File} $\rightarrow$ \textit{Settings \ldots}
\item Select \textit{Plugins} in the left pane
\item Click \textit{Install plugin from disk...}
\item Select a JAR or ZIP file containing the plugin
\end{enumerate}

\section{Metamodel}
TODO: INSERT DIAGRAM

TODO: INSERT EXAMPLE QUESTIONNAIRE WITH ANNOTATIONS

Our main questionnaire is a \textbf{Form}. The form contains a single \textbf{Block}. This block in turn contains a set of \textit{FormElement}s. 

We distinguish several types of form elements:
\begin{description}
\item[Question] A value provided by a user filling in the questionnaire. Questions are identified by a name and have a type.
\item[CalculatedValue] A value that is derived (calculated) from other values on the form. Calculated values are identified by name and have a type. Additionally they have an expression that defines how the value should be calculated. This expression references other questions and calculated values.
\item[Block] A container that can hold other form elements.
\item[ConditionalBlock] Some parts of the form should only be displayed when certain conditions are met, these parts are under a ConditionalBlock. Conditional blocks have a condition (or 
\item[FormElement] An ``empty'' form element which serves as a blank spot in the questionnaire. These blanks can be filled in by substituting a ``real'' form element.
\end{description}

As mentioned before, Questions and calculated values have a \textbf{Type}. This type can be \textbf{Boolean}, \textbf{Integer} or \textbf{String}.



introduce concepts
structure definition

\section{Editor}
simple editor for the concepts

\section{Generator}
options (java, html, etc.) - decide what to use
assumptions to make things easier

just let say the program say hello

actually updating questions

determining update order

saving/reading output

Full case (QLRuntime factored out)

\section{Advancing the Editor}

% appendices or something like that
\section{QLS}

\section{SAT checking}

\end{document}
